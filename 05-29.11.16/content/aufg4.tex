\section*{Aufgabe 4)}
\subsection*{a)}
\textbf{Lineare (Fisher) Diskriminanzanalyse:} \\
Die LDA soll zwei Klassen mit $n$ Observablen durch eine (n-1)-dim Hyperebene optimal trennen. Wird zum Beispiel bei der Trennung von einem Rauschen und eines Signal verwendet. Dadurch wird das Signal besser erkennbar bzw. überhaupt erkennbar. \\
\textbf{Feature Extraction:} \\
Die Feature Extraction soll mehrere Attribute zusammenfassen oder transformieren und daraus neue Attribute erstellen. Beipielsweise können ringförmig angeordnete Verteilungen, mit Hilfe von Polarkoordinaten in eine lineare Verteilung transformiert werden. \\
\textbf{Feature Selection:} \\
Die Feature Selection verwirft vorhandene Attribute um ein Subset zu finden. Dafür wird zum Beispiel der "Brute Force"-Algorithmus verwendet, welcher durch einfaches ausprobieren aller Möglichkeiten das beste Ergebnis findet(nicht praktikabel). Die Forward-/Backward-Selection ist ein iteratives Verfahren, welches entweder von vorne oder hinten durch alle Messwerte durchläuft und erst bei einer angegeben Qualität stopt. \\
\\
Außerdem müssen alle missverständlichen Werte, wie NaN oder Inf, durch bessere Werte ersetzt oder entfernt werden. Zusätzlich dazu müssen konstante Attribute oder Attribute mit zu vielen fehlenden Werten entfernt werden.



\subsection*{b)}
Eine Normierung ist Sinnvoll. \\
Bei der LDA werden normierte Punkte zum Beispiel orthogonal auf die Hyperebene abgebildet, nicht normierte Punkte werden nicht unbedingt orthogonal abgebildet.



\subsection*{c)}
Lücken in den Daten werden durch die Mittelwerte der Verteilung aufgefüllt.



\subsection*{d)}
Die Datensätze müssen kombinierbar sein, dafür müssen sie die gleichen Attribute haben und gleich viele Einträge besitzten.
