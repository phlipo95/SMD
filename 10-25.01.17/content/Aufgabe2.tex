\section*{Aufgabe 2}
Mittels den beigelegtem Python Skript werden die $\chi^2$ Werte anhand von Formel 
\begin{equation}
  \chi^2 = \sum_{i = 1}^n \frac{(y_i-<y_t(x_i)>)^2}{\sigma^2_{t,i}}
\end{equation}
berechnet.
\subsection*{a)}
Die Anzahl an Freiheitsgraden entspricht der Anzahl an Messwerten - 1. Es ergibt sich ein $\chi^2$ Wert von
\begin{equation}
  58.5 \ .
\end{equation}
Daraus lässt sich aus der $\chi^2$ Tabelle der Vorlesung eine $\alpha$ von weniger als 0.01 abgelesen. Daraus lässt sich schließen das der Wert nicht verworfen werden muss. 
\subsection*{b)}
Es wird ein $\chi^2$ Wert von
\begin{equation}
  9.54
\end{equation} 
berechnet. Daraus lässt sich anhand der Tabelle ein $\alpha$ zwischen in etwa $~0.25$ ablesen. Damit muss der Wert bei einer Signifikanz von 5 \% verworfen werden. 
