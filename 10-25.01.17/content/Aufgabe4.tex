\section*{Aufgabe 4}
Die Wahrscheinlichkeiten, dass Kaonen, Pionen und Protonen beobachtet wurden, werden mit dem Bayes Theorem berechnet:
\begin{align*}
	p(H_i|D,I) &= \frac{p(H_i|I) p(D|H_i,I)}{p(D|I)}
\end{align*}
Die Wahrscheinlichkeiten $p(D|H_i,I)$ die Daten $D$ zu messen, sind aus den Likelihoods gegeben. \\
Die Prior-Informtionen sind gegeben mit
\begin{align*}
	p_\Pi(H_1|I) &= 0,8	\\
	p_K(H_2|I) &= 0,1	\\
	p_p(H_3|I) &= 0,1
\end{align*}
Die Normierung des zu berechnenden Posteriors $p(D|I)$ ist gegeben mit:
\begin{align*}
	p(D|I) &= \sum\limits_i p(H_i|I) p(D|H_i)
\end{align*}

\begin{itemize}
	\item[a)]
		Gegeben sind:
		\begin{align*}
			p_\Pi(D|H_1,I) &= 0,13	\\
			p_K(D|H_2,I) &= 1,5	\\
			p_p(D|H_3,I) &= 0,5
		\end{align*}
		Die Prior-Information ergibt sich zu:
		\begin{align*}
			p(D|I) &= 0,8 \cdot 0,13 + 0,1 \cdot 1,5  + 0,1 \cdot 0,5	\\
				&= 0,304
		\end{align*}
		Daraus ergeben sich die Wahrscheinlichkeiten zu:
		\begin{align*}
			p_\Pi(H_1|D,I) &= \frac{0,8 \cdot 0,13}{0,304} \approx 0,3421 = \SI{34,21}{\percent}	\\
			p_K(H_2|D,I) &= \frac{0,1 \cdot 1,5}{0,304} \approx 0,4934 = \SI{49,34}{\percent}	\\	
			p_p(H_3|D,I) &= \frac{0,1 \cdot 0,5}{0,304} \approx 0,1645 = \SI{16,45}{\percent}
		\end{align*}
	
	\item[b)]
		Gegeben sind:
		\begin{align*}
			p_\Pi(D|H_1,I) &= 2,0	\\
			p_K(D|H_2,I) &= 0,5	\\
			p_p(D|H_3,I) &= 0,05
		\end{align*}
		Die Prior-Information ergibt sich zu:
		\begin{align*}
			p(D|I) &= 1,655
		\end{align*}
		Daraus ergeben sich die Wahrscheinlichkeiten zu:
		\begin{align*}
			p_\Pi(H_1|D,I) & = \SI{96,68}{\percent}	\\
			p_K(H_2|D,I) & =  \SI{3,02}{\percent}	\\	
			p_p(H_3|D,I) &=  \SI{0,30}{\percent}
		\end{align*}
	
		
	\item[c)]
		Gegeben sind:
		\begin{align*}
			p_\Pi(D|H_1,I) &= 0,07	\\
			p_K(D|H_2,I) &= 0,5	\\
			p_p(D|H_3,I) &= 1,3
		\end{align*}
		Die Prior-Information ergibt sich zu:
		\begin{align*}
			p(D|I) &= 0,236
		\end{align*}
		Daraus ergeben sich die Wahrscheinlichkeiten zu:
		\begin{align*}
			p_\Pi(H_1|D,I) & = \SI{23,73}{\percent}	\\
			p_K(H_2|D,I) & = \SI{21,19}{\percent}	\\	
			p_p(H_3|D,I) &= \SI{55,08}{\percent}
		\end{align*}
	
\end{itemize}