\section*{Aufgabe 1:}

\begin{figure}
  \centering
  \includegraphics[height=10cm]{build/Aufgabe1.pdf}
\end{figure}

\begin{align*}
	f(x) &= (1-x)^6 \\
	g(x) &= x^6 -6x^5 +15x^4 -20x^3 +15x^2 -6x +1 \\
	h(x) &= x( x( x( x( (x-6)x +15) -20) +15) -6) +1
\end{align*}

\newpage

\section*{Aufgabe 2:}
\subsection*{Aufgabe 2 a:}

\begin{align*}
	& \lim\limits_{x \to 0} \frac{\sqrt{9-x}-3}{x} \\
	\overset{\text{l'Hopital}}{=}& \lim\limits_{x \to 0} \frac{\frac{d}{dx}(\sqrt{9-x}-3)}{\frac{d}{dx}x} \\
	=& \lim\limits_{x \to 0} \frac{-1}{2\sqrt{9-x}} = - \frac{1}{6}
\end{align*}

\subsection*{Aufgabe 2 b:}

\begin{table}
  \centering
  \begin{tabular}{c|c}
		x & f(x) = $\frac{\sqrt{(9-x)}-3}{x}$ \\
		\hline
		$10^{-1}$  & -0.16713222 \\
		$10^{-2}$  & -0.16671299 \\
		$10^{-3}$  & -0.16667130 \\
		$10^{-4}$  & -0.16666713 \\
		$10^{-5}$  & -0.16666671 \\
		$10^{-6}$  & -0.16666667 \\
		$10^{-7}$  & -0.16666667 \\
		$10^{-8}$  & -0.16666668 \\
		$10^{-9}$  & -0.16666668 \\
		$10^{-10}$ & -0.16666668 \\
		$10^{-11}$ & -0.16666668 \\
		$10^{-12}$ & -0.16653345 \\
		$10^{-13}$ & -0.16431301 \\
		$10^{-14}$ & -0.17763568 \\
		$10^{-15}$ & -0.44408921 \\
		$10^{-16}$ & 	0.0 \\
		$10^{-17}$ & 	0.0 \\
		$10^{-18}$ &  0.0 \\
		$10^{-19}$ &  0.0 \\
		$10^{-20}$ &  0.0 \\
  \end{tabular}
\end{table}

Wenn das x zu klein wird, wird 9-x gegen 9 genähert. Dadurch steht im Zähler gleich 0 und das Ergebnis ist auch gleich 0.

\newpage
\section*{Aufgabe 3:}

\begin{align*}
	f(x) = (x³ + \frac{1}{3}) - (x³ - \frac{1}{3})
\end{align*}

In einem Intervall von 0 $\le$ x $\le$ 41300 weicht f(x) um weniger als 1\% vom Theoriewert ab. \\
In einem Intervall von 0 $\le$ x $\le$ 0.5504 weicht f(x) um 0\% vom Theoriewert ab. \\

\begin{align*}
	g(x) = \frac{(3 + \frac{x³}{3}) - (3 - \frac{x³}{3})}{x³}
\end{align*}

In einem Intervall von $10^{-6} \le$ x $\le 10^{100}$ weicht f(x) um weniger als 1\% vom Theoriewert ab. \\

\begin{figure}
	\centering
	\includegraphics{build/Aufgabe3.pdf}
\end{figure}
























%
