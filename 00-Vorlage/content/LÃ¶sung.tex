\section*{Aufgabe 1:}

\begin{figure}
  \centering
  \includegraphics[height=10cm]{build/Aufgabe1.pdf}
\end{figure}

\begin{align*}
	f(x) &= (1-x)^6 \\
	g(x) &= x^6 -6x^5 +15x^4 -20x^3 +15x^2 -6x +1 \\
	h(x) &= x( x( x( x( (x-6)x +15) -20) +15) -6) +1
\end{align*}

\newpage

\section*{Aufgabe 2:}
\subsection*{Aufgabe 2 a:}

\begin{align*}
	& \lim\limits_{x \to 0} \frac{\sqrt{9-x}-3}{x} \\
	\overset{\text{l'Hopital}}{=}& \lim\limits_{x \to 0} \frac{\frac{d}{dx}(\sqrt{9-x}-3)}{\frac{d}{dx}x} \\
	=& \lim\limits_{x \to 0} \frac{-1}{2\sqrt{9-x}} = - \frac{1}{6}
\end{align*}

\subsection*{Aufgabe 2 b:}

\begin{table}
  \centering
  \begin{tabular}{c|c}
		x & f(x) = $\frac{\sqrt{(9-x)}-3}{x}$ \\
		\hline
		$10^{-1}$  & -0.16713222 \\
		$10^{-2}$  & -0.16671299 \\
		$10^{-3}$  & -0.16667130 \\
		$10^{-4}$  & -0.16666713 \\
		$10^{-5}$  & -0.16666671 \\
		$10^{-6}$  & -0.16666667 \\
		$10^{-7}$  & -0.16666667 \\
		$10^{-8}$  & -0.16666668 \\
		$10^{-9}$  & -0.16666668 \\
		$10^{-10}$ & -0.16666668 \\
		$10^{-11}$ & -0.16666668 \\
		$10^{-12}$ & -0.16653345 \\
		$10^{-13}$ & -0.16431301 \\
		$10^{-14}$ & -0.17763568 \\
		$10^{-15}$ & -0.44408921 \\
		$10^{-16}$ & 	0.0 \\
		$10^{-17}$ & 	0.0 \\
		$10^{-18}$ &  0.0 \\
		$10^{-19}$ &  0.0 \\
		$10^{-20}$ &  0.0 \\
  \end{tabular}
\end{table}

Wenn das x zu klein wird, wird 9-x gegen 9 genähert. Dadurch steht im Zähler gleich 0 und das Ergebnis ist auch gleich 0.

\newpage
\section*{Aufgabe 3:}

\begin{align*}
	f(x) = (x³ + \frac{1}{3}) - (x³ - \frac{1}{3})
\end{align*}

In einem Intervall von 0 $\le$ x $\le$ 41300 weicht f(x) um weniger als 1\% vom Theoriewert ab. \\
In einem Intervall von 0 $\le$ x $\le$ 0.5504 weicht f(x) um 0\% vom Theoriewert ab. \\

\begin{align*}
	g(x) = \frac{(3 + \frac{x³}{3}) - (3 - \frac{x³}{3})}{x³}
\end{align*}

In einem Intervall von $10^{-6} \le$ x $\le 10^{100}$ weicht f(x) um weniger als 1\% vom Theoriewert ab. \\

\begin{figure}
	\centering
	\includegraphics{build/Aufgabe3.pdf}
\end{figure}
\section*{Aufgabe 4}
\subsection*{a) Numerische Stabilität des Differentiellen Wirkungsquerschnitts}
Die Gleichung $\frac{d\Rho}{d\Omega}$ ist numerisch instabil sobald im Nenner eine zu kleine Zahl steht. Strebt die Funktion 
\begin{equation}
  \lim\limits_{\theta \to n\pi} \frac{\alpha^2}{s} \left( \frac{2+\sin^2{\theta}}{1-(1-(m_\text{e}/E_\text{e})^2) \cos^2(\theta)} \right)
  \label{eqn:wirk}
\end{equation}
gegen ein ganzzahliges vielfache von $\pi$ wird $\cos(\theta)^2$ =  1 und der Nenner strebt gegen den Wert $\frac{m_\text{e}^2}{E_\text{e}^2}$ welcher in der Größenordnung $10^{-10}$. Aufgrund der Division durch eine kleine Zahl ist der Term an den vielfachen von $\pi$ schlecht konditioniert und daher auch numerisch Instabil.
\subsection*{b) Stabilisierung des Numerischen Problems}
Mit Hilfe des Hinweise lässt sich der Numerische Ausdruck zu 
\begin{eqnarray}
  1 -\beta^2 \cos(\theta)^2 =& \sin(\theta)^2 + \cos(\theta)^2 -\beta^2 \cos(\theta)^2 \\
  \sin(\theta)^2 + (1 - \beta^2) \cos(\theta)^2 =& sin(\theta)^2 + \frac{1}{\gamma^2} \cos(\theta)^2 \\
 =& sin^2(\theta) + \left( \frac{m_\text{e}}{E_\text{e}} \right)^2 \cos^2(\theta) \\
  \label{}
\end{eqnarray}
und
\begin{equation}
  2+\sin(\theta)^2 = 3 - \cos(\theta)^2
  \label{<++>}
\end{equation}
umformen. Daraus ergibt sich ein stabilisierte Term von 
\begin{equation}
  \frac{3 - \cos(\theta)^2}{\sin(\theta)^2 + (\frac{m_\text{e}}{E_\text{e}})^2 \cos(\theta)^2} \ .
  \label{}
\end{equation}
\subsection*{c)}
\begin{figure}
  \centering
  \includegraphics[height=6cm]{build/Aufgabe4.pdf}
  \caption{Vergleich der beiden Graphen}
  \label{fig:Vergleich der beiden Graphen}
\end{figure}
\subsection*{d) Konditionszahl}
Die Konditionszahl ist definiert als 
\begin{equation}
  \kappa(x) = \left| x\frac{f'(x)}{f(x)} \right|
  \label{}
\end{equation}
für den nicht stabelisierten Term ergibt sich 
\begin{eqnarray}
  \frac{df(\theta)}{d\theta} = \frac{\alpha^2}{s} \left( \frac{2\sin(\theta)\cos(\theta)}{1 - \beta^2 \cos(\theta)^2} - \frac{(2+\sin(\theta)^2)2\beta^2\sin(\theta)\cos(\theta)}{(1-\beta^2\cos(\theta)^2)^2} \right)
  \label{}
\end{eqnarray}
Kappa berechnet sich somit zu 
\begin{equation}
  \kappa(\theta) = \left| \theta \left( \frac{df(\theta)}{d\theta} \right) \left( \frac{d\sigma}{d\Omega} \right) \right| = \left| \theta \left( \frac{2\sin(\theta)\cos(\theta)}{2 + \sin(\theta)^2} - \frac{2\beta^2\sin(\theta)\cos(\theta)}{1 - \beta^2 \cos(\theta)^2} \right) \right|
  \label{}
\end{equation}
Für den stabelisierten Term lautet die Ableitung
\begin{equation}
  \frac{df(\theta)}{d\theta} = \frac{\alpha^2}{s}\left( \frac{2\sin(\theta)\cos(\theta)}{sin(\theta)^2+\gamma^{-2}\cos(\theta)^2} + \frac{(3-\cos(\theta)^2)(2\gamma^{-2}\sin{\theta}\cos{\theta}-2\sin(\theta)\cos(\theta))}{(\sin(\theta)^2 + \gamma^{-2}\cos(\theta)^2)} \right)
  \label{<++>}
\end{equation}
Es ergibt sich ein Kappa von 
\begin{eqnarray}
  \kappa(\theta) =& \left| \theta \left( \frac{2\sin(\theta)\cos(\theta)}{3 - \cis(\theta)^2} + \frac{2\gamma^{-2}\sin(\theta)\cos(\theta) - 2\sin(\theta)\cos(\theta)}{\sin(\theta)^2 + \gamma^{-2}\cos(\theta)^2}  \right) \right| \\
  =& \left| \theta \left( \frac{2\sin(\theta)\cos(\theta)}{3 - \cos(\theta)^2} - \frac{2\beta^2\sin(\theta)\cos(\theta)}{\sin(\theta)^2 + \gamma^2 \cos(\theta)^2} \right) \right|
\end{eqnarray}.
\subsection*{e)}
\begin{figure}
  \centering
  \includegraphics{build/Aufgabe4e.pdf}
  \caption{Verlauf der Konditionszahl}
  \label{fig:<+label+>}
\end{figure}
