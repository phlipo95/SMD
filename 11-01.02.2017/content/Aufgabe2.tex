\subsection*{a)}
Die Antwortmatrix $A$ auf den Sachverhalt das bei dem Experiment 20 \% der Messung verloren geht hat die Form
\begin{equation}
  A = 0.8 \cdot  
  \begin{pmatrix}
    1 & 0 \\
    0 & 1 \\
  \end{pmatrix} \ .
\end{equation}
$\varepsilon$ ist die Wahrscheinlichkeit, dass das Ergebnis dem falschen Bin zugeordnet wird. Daraus ergibt sich die Antwortmatrix
\begin{equation}
  0.8 \cdot
  \begin{pmatrix}
    1 - \varepsilon & \varepsilon \\
    \varepsilon & 1 - \varepsilon \\
  \end{pmatrix}
\end{equation}
Anhand derer lässt sich aus den wahren Ereignisszahlen $f = (f_1, f_2)^T$ auf die gemessenen Ereigniszahlen $g = (g_1,g_2)^T$ schließen.
\begin{equation}
  g = A \cdot f \ 
\end{equation}
\subsection*{b)}
Um aus den gemessenen Ereignisszahlen $g$ auf die muss die wahren $f$ zu schließen wird zunächst die Inverse der Antwortmatrix ($=B$) berechnet.
\begin{equation}
  B = \frac{0.8}{0.8^2 \cdot (1 - 2 \varepsilon)} 
  \begin{pmatrix}
    1 - \varepsilon & - \varepsilon \\
    - \varepsilon & 1 - \varepsilon \\
  \end{pmatrix}
\end{equation}
Aus ihr lässt sich die wahren Ereignisszahlen $f$ als Funktion von $\varepsilon$ und $f$ berechnen. 
\begin{equation}
  f = B * g
\end{equation}
Daraus lassen sich die Einträge von 
\begin{equation}
  f_1 = \frac{1}{0.8 - 1.6 \varepsilon} \left( (1 - \varepsilon) g_1 - \varepsilon g_2 \right)
\end{equation}
und 
\begin{equation}
  f_2 = \frac{1}{0.8 - 1.6 \varepsilon} \left( (1 - \varepsilon) g_2 - \varepsilon g_1 \right)
\end{equation} 
ablesen. 
\subsection*{c)}
Aus dem Vektor $g$ wird die Kovarianzmatrix berechnet indem die gemessenen Einträgte  der Intervalle auf die Hauptdiagonale geschrieben werden,
\begin{equation}
  V[g] =   
  \begin{pmatrix}
    g_1 & 0 \\
    0 & g_2 \\
  \end{pmatrix}
\end{equation}
und anschließend in die Basis von f transformiert 
\begin{equation}
  V[f] = B V[g] B^\text{T} = \frac{1}{(0.8 - 1.6 \varepsilon)^2}
  \begin{pmatrix}
    1 - \varepsilon & - \varepsilon \\
    - \varepsilon & 1 - \varepsilon \\
  \end{pmatrix}
  \begin{pmatrix}
    g_1 & 0 \\
    0 & g_2 \\
  \end{pmatrix}
  \begin{pmatrix}
    1 - \varepsilon & - \varepsilon \\
    - \varepsilon & 1 - \varepsilon \\
  \end{pmatrix}
\end{equation}
\begin{equation}
  \Leftrightarrow \frac{1}{(0.8 - 1.6 \varepsilon)^2}
  \begin{pmatrix}
    (1 - \varepsilon)^2 g_1 + \varepsilon^2 g_2 & -\varepsilon(1-\varepsilon)(g_1 +g_2) \\
    -\varepsilon(1-\varepsilon)(g_1 +g_2) & (1 - \varepsilon)^2 g_2 + \varepsilon^2 g_1  \\
  \end{pmatrix}
\end{equation}
\subsection*{d)}
Die Fehler der Messwerte werden aus der Wurzel der Diagonalelemente berechnet. Der Rest wird aus dem entsprechenden Pythoncode berechnet.
\begin{equation}
  f = \begin{pmatrix}
    \num{254 +- 20}\\
    \num{206 +- 18}\\
    \end{pmatrix}
\end{equation}
Die Kovarianzmatrix ist :
\begin{equation}
  \begin{pmatrix}
    399 & -81 \\
    -81 & 339 \\
  \end{pmatrix}
\end{equation}

\subsection*{e)}
Die Fehler der wahren Ereignisszahlen werden aus der Wurzel der Diagonalelemente berechnet.
\begin{equation}
  f = \begin{pmatrix}
    \num{327 +- 62}\\
    \num{133 +- 60}\\
    \end{pmatrix}
\end{equation}
Die Kovarianzmatrix ist :
\begin{equation}
  \begin{pmatrix}
    584 & -256 \\
    -256 & 584 \\
  \end{pmatrix}
\end{equation}

\subsection*{f)}
Für $\varepsilon$ = 0.5 wird die Matrix singulär, sodass die Inverse Matrix nicht zu bestimmen ist.
